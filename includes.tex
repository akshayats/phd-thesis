\usepackage[T1]{fontenc}
\usepackage{textcomp}
\usepackage{lmodern}
%\usepackage[latin1]{inputenc}
\usepackage[utf8]{inputenc}
\usepackage[swedish,english,romanian]{babel}
\usepackage{combelow}
\usepackage{xargs}                      % Use more than one optional 
\usepackage[pdftex,dvipsnames]{xcolor}  % Coloured text etc.
\usepackage{pdflscape}

% Use this to make things fit in less space
%\renewcommand{\baselinestretch}{0.988}


\usepackage{xcolor,colortbl}
\usepackage{graphicx}
\usepackage{graphics} % for pdf, bitmapped graphics files
\usepackage{epsfig} % for postscript graphics files
\usepackage{amsmath} % assumes amsmath package installed
\usepackage{amssymb}  % assumes amsmath package installed
\usepackage{color}
\usepackage{subfig}
\usepackage[nocompress]{cite}
\usepackage{algorithm}
\usepackage{tikz}
\usepackage{booktabs}
\usepackage{pgffor}
\usepackage{alphalph}
\usepackage{cite}
\usepackage{soul}
\definecolor{Gray}{gray}{0.6}
\definecolor{SoftGray}{gray}{0.75}
\definecolor{SoftestGray}{gray}{0.9}
\newcolumntype{a}{>{\columncolor{Gray}}r}
\newcolumntype{b}{>{\columncolor{SoftGray}}r}
\newcolumntype{d}{>{\columncolor{SoftestGray}}r}
\renewcommand\thesubfigure{\alphalph{\value{subfigure}}}
\usetikzlibrary{arrows}

\usepackage{mathptmx} % assumes new font selection scheme installed
\usepackage{times} % assumes new font selection scheme installed
\usepackage{caption}
\usepackage{algorithm}
\usepackage{algpseudocode}

%\usepackage[pdftex]{graphicx}
%\usepackage[usenames,dvips,pdftex]{color}
\usepackage{wrapfig}
\usepackage{longtable}
\usepackage{wrapfig}
\usepackage{units}
\usepackage{url}
\usepackage[normalem]{ulem}
\usepackage{pdfpages}
\usepackage{epstopdf}
\usepackage{array,multirow}
\usepackage{hhline}
\usepackage{stackengine}
\usepackage[percent]{overpic}
\usepackage{rotating}
\usepackage{tabularx}
\usetikzlibrary{calc,arrows}
\usepackage{gensymb} %degree symblo
\usepackage{setspace}
\usepackage{parskip}

% hyperlinks
\usepackage{hyperref}
\hypersetup{
    colorlinks=false, %set true if you want colored links
    linktoc=all,     %set to all if you want both sections and subsections linked
    linkcolor=black,  %choose some color if you want links to stand out
}


% TODO shortcut commands 
\usepackage[colorinlistoftodos,prependcaption,textsize=tiny]{todonotes}
%\usepackage[disable]{todonotes}
\newcommandx{\unsure}[2][1=]{\todo[linecolor=red,backgroundcolor=red!25,bordercolor=red,#1]{#2}}
\newcommandx{\change}[2][1=]{\todo[linecolor=blue,backgroundcolor=blue!25,bordercolor=blue,#1]{#2}}
\newcommandx{\info}[2][1=]{\todo[linecolor=OliveGreen,backgroundcolor=OliveGreen!25,bordercolor=OliveGreen,#1]{#2}}
\newcommandx{\improvement}[2][1=]{\todo[linecolor=Plum,backgroundcolor=Plum!25,bordercolor=Plum,#1]{#2}}
\newcommandx{\thiswillnotshow}[2][1=]{\todo[disable,#1]{#2}}





% Metaroom includes

\newsavebox\IBoxA \newsavebox\IBoxB \newlength\IHeight
\newcommand\TwoFig[8]{% Image1 Caption1 Label1 Image2 ...
  \sbox\IBoxA{\includegraphics[width=#7\textwidth]{#1}}
  \sbox\IBoxB{\includegraphics[width=#8\textwidth]{#4}}%
  \ifdim\ht\IBoxA>\ht\IBoxB
    \setlength\IHeight{\ht\IBoxB}\else\setlength\IHeight{\ht\IBoxA}\fi%
  \begin{figure}[!htb]
  \minipage[t]{0.2\textwidth}\centering
  \includegraphics[height=\IHeight]{#1}
  \caption{#2}\label{#3}
  \endminipage\hfill
  \minipage[t]{0.2\textwidth}\centering
  \includegraphics[height=\IHeight]{#4}
  \caption{#5}\label{#6}
  \endminipage 
  \end{figure}%
}

\newcommand{\pluseq}{\mathrel{+}=}
\newcommand{\minuseq}{\mathrel{-}=}

\DeclareMathOperator{\Div}{div}
\DeclareMathOperator{\grad}{grad}
\newcommand{\matr}[1]{\boldsymbol{#1}}	%Matrices and Vectors
\newcommand{\compl}[1]{\underline{#1}} 	%Complex Variables
\newcommand{\E}{\mbox{I\negthinspace E}}	%Expectation
\renewcommand{\Re}{\operatorname{Re}}	%reel value
\renewcommand{\Im}{\operatorname{Im}}		%imaginary value
\newcommand{\avg}[1]{\left< #1 \right>} % for average
\newcommand{\myUnit}[1]{\ensuremath{\mathrm{#1}}}
%\newcommand{\myUnit}[1]{\mathrm{#1}}
\DeclareMathOperator{\e}{e} %exponential
\DeclareMathOperator*{\argmin}{arg\,min}
\DeclareMathOperator*{\argmax}{arg\,max}
\newcommand{\mycite}[1]{\begin{small}\textcolor{gray}{#1}\end{small}}
\newcommand\norm[1]{\left\lVert#1\right\rVert}


\newcommand{\cblue}[1]{\textcolor{blue}{#1}}
\newcommand{\cred}[1]{#1}
\newcommand\tikzmark[2]{ \tikz[remember picture,overlay]  \node[inner sep=0pt,outer sep=2pt] (#1){#2};%
}

\newcommand\link[2]{%
\begin{tikzpicture}[remember picture, overlay, >=stealth, shorten >= 1pt]
  \draw[->] (#1.east) to  (#2.west);
\end{tikzpicture}%
}

\maxsecnumdepth{subsection}
\setsecnumdepth{subsection}
% Uncomment this to add subsections to TOC
%\maxtocdepth{subsection}
%\settocdepth{subsection}

% Sans serif fonts in chapter and section headings can look nice.
\renewcommand{\partnamefont} {\usefont{T1}{lmss}{sbc}{n}\boldmath\huge}
\renewcommand{\partnumfont} {\usefont{T1}{lmss}{sbc}{n}\boldmath\huge}
\renewcommand{\parttitlefont}{\usefont{T1}{lmss}{sbc}{n}\boldmath\Huge}
\renewcommand{\chapnamefont} {\usefont{T1}{lmss}{sbc}{n}\boldmath\huge}
\renewcommand{\chapnumfont} {\usefont{T1}{lmss}{sbc}{n}\boldmath\huge}
\renewcommand{\chaptitlefont}{\usefont{T1}{lmss}{sbc}{n}\boldmath\Huge}
\setsecheadstyle {\usefont{T1}{lmss}{bx}{n}\boldmath\Large\raggedright}
\setsubsecheadstyle{\usefont{T1}{lmss}{bx}{n}\boldmath\large\raggedright}
\setparaheadstyle {\normalsize\bfseries\boldmath}
\setsubparaheadstyle{\normalsize\bfseries\boldmath}

% Instead of slanted type in the page headers, use upright, slightly smaller, type.
\makeevenhead{headings}%
{\normalfont\small\thepage}{}{\normalfont\small\leftmark}
\makeoddhead{headings}%
{\normalfont\small\rightmark}{}{\normalfont\small\thepage}

%We also modify the style of figure and table captions slightly.
%\setlength{\@tempdima}{\textwidth}
%\addtolength{\@tempdima}{-2\leftmargini}
%\captionwidth{\@tempdima}
%\changecaptionwidth
\captiondelim{. }
\captionnamefont{\normalfont\footnotesize\bfseries}
\captiontitlefont{\normalfont\footnotesize}

%For some reason, standard LATEX has glue with non-zero stretchability in \parskip. Not nice. . .
\setlength{\parskip}{0pt}

% Thippur Custom Commands
% -----------------------
\newcommand*{\pretoctitle}[1]{{\clearpage\centering
\vspace*{-0.5cm}\textbf{#1}\vspace{0.5cm}\par}}