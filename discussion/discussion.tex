
\chapter{Final Discussion}
\label{chap:discussion}
\renewcommand{\chapterheadstart}{~\vspace{1.7cm}\\}

Discussion


% Include in introduction section, file
\section{Contextual learning - general discussion}
\begin{itemize}
\item Saikat - What is it learning to learn? Dog example
\item Kid learns about dogs only by cartoons - small representative subset of dogs. Then when exposed to various dogs in a park, there is not a single error type 1 or type 2 of detecting dogs. Why?
\item Generalisability with no common examples. No supervision either of the generalising nature. No guidelines even by explicit explanations. Not even possibly a one shot learning!
\item the kid learnt by context of activity, the context of environment and spatial conformation of sub objects such as leash, collar, stick or ball in the mouth?
\item When is an aritficial agent intelligent?
\begin{itemize}
\item Test by hardest example accuracy to the easiest. does performance decrease?
\item Robustness to Noise
\item Can maintain robustness across multiconditioning tests, simultaneously and individually tested.
\end{itemize}
\item Fruits and veggies behind plastic sheets, fails because of lack of contextual support or even biased contextual knowledge. Most visual illusions work this way. Class boundary specification needs to be explicit, otherwise what is in our head needs to map to the machine through data examples! We as humans know "what happens when there is plastic sheets interfering with vision?" The machine does not.
\item Unrelated: What is Selection bias? Ping pong theorem?
\end{itemize}
